\documentclass{article}
\usepackage[utf8]{inputenc}

\title{Online voting: measure of privacy and verifiability}
\author{boire.sebastien }
\date{November 2020}

\usepackage{natbib}
\usepackage{graphicx}
\usepackage{amsmath}
\usepackage{amssymb}

\begin{document}

\maketitle


\section{First scheme: public storage of a subset of possible vote values}

\subsection{Description}

We consider an election with candidates $C_1, ..., C_K$. Voter V votes for $C_i$, and the value of the vote is stored encrypted so that it can be counted, verified by V, and secret. In addition, (k-1) other candidates are chosen at random, and a public ticket is produced: "Voter V voted for $C_{i_1}$ or $C_{i_2}$ or ... or $C_{i_k}$.", with one of the $C_{i_j}$ being $C_i$, the correct value.

\subsection{Evaluation of verifiability}

We note $X^i_j$ the binary variable corresponding to "Does voter j voted for candidate i?". Then:

$Y_\alpha^i=X^i_1+...+X^i_\alpha$ is a binomial variable following $\mathcal{B}(\alpha, p_i)$, with $p_i$ probability of vote for candidate i.

In a similar way, we note $\tilde{X}^i_j$ the binary variable corresponding to "Does the public vote of voter j contains candidate i as a possible vote?".

$\tilde{Y}_\alpha^i=\tilde{X}^i_1+...+\tilde{X}^i_\alpha$ is a binomial variable following $\mathcal{B}(\alpha, \tilde{p}_i)$, with $\tilde{p}_i$ probability that candidate i is in a public vote.

We have:

$\tilde{p}_i=p_i+(1-p_i)*\frac{k-1}{K-1}$


We use Bienayme-Tchebychev inequality on $\tilde{Y}_\alpha^i$:

$ \mathbb{P}(|\dfrac{\tilde{Y}^i_\alpha-\alpha*\tilde{p}_i}{\alpha}| > x )  \leqslant \dfrac{\tilde{p}_i(1-\tilde{p}_i)}{\alpha x^2}$


Using the public votes of $\alpha$ voters, we can compare the scores we obtain for each candidate to the global results of the election, and measure whether the difference is likely or not. This provides a measure of verifiability (we just need to choose a level of verification with the term $\dfrac{\tilde{p}_i(1-\tilde{p}_i)}{\alpha x^2}$).


\subsection{Evaluation of privacy}

Privacy can be defined in multiple ways. Here, we will measure the entropy on one vote, as the amount of unknown information to an attacker to know one vote.

$Privacy=H(vote)=-\sum\limits_{c \in candidates} p(c)log_2(p(c))=log_2(k)$

In this, we do not take into account that each candidate has a probability $p_i$ of being the real value of the vote (to simplify calculations). This corresponds to the situation where each candidate obtains as much vote as any other.


\section{Second scheme: public storage of the votes under probabilistic value}


\subsection{Description}

The election is in the same configuration as in the previous section. However, before the the voter chooses who he votes for, he sees a set of random variables appearing on the screen: $R_1, ..., R_K$. They all follow a Gaussian law, with $R_1$ following a $\mathcal{N}(\mu, \sigma^2)$, and all the other $R_i$ following a $\mathcal{N}(\frac{1-\mu}{K}, \sigma^2)$. Then, the voter chooses who he votes for, and the variables $R_i$ are rotated so that $R_1$ is associated to the candidate chosen by the voter. The values of the $R_i$ are then made public. 

$\mu$ is chosen such that $\mu > \frac{1-\mu}{K}$, so that on average we can distinguish the correct candidate. The bigger the difference, the easier it is to identify the real value of the vote. In addition, $\sigma$ quantifies the randomness of the values of $R_i$, and also takes part in how easy it is to identify the real value of a vote.

\bigbreak

This structure allows the voter to be sure that he is not duped on the value of his public vote, since the random variables $R_i$ are evaluated before the voter enters his vote.


\subsection{Verifiability evaluation}



\subsection{Privacy evaluation}


\section{Third scheme: Privacy of the votes depending on subgroups of the voting population}

\end{document}
 
